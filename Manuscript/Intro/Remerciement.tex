%%%%%%%%%%%%%%%%%%%%%%%%%%%%%%%%%%%%%%%%%%%%%%%%%%%%%%%%%%%%%%%%%%%%%%%%%%
%%%%%                        Remerciements                          %%%%%%
%%%%%%%%%%%%%%%%%%%%%%%%%%%%%%%%%%%%%%%%%%%%%%%%%%%%%%%%%%%%%%%%%%%%%%%%%%

\phantomsection 
\addcontentsline{toc}{section}{Acknowledgements}
\addtocontents{toc}{\protect\addvspace{5pt}}

\vspace*{-1.6cm}
\begin{flushright}
\section*{\fontsize{20pt}{20pt}\selectfont\textnormal{Acknowledgements}}
\end{flushright}
\vspace{-0.2cm}


\lhead[\fancyplain{}{Acknowledgements}]
      {\fancyplain{}{}}
\chead[\fancyplain{}{}]
      {\fancyplain{}{}}
\rhead[\fancyplain{}{}]
      {\fancyplain{}{Acknowledgements}}
\lfoot[\fancyplain{}{}]
      {\fancyplain{}{}}
\cfoot[\fancyplain{}{\thepage}]
      {\fancyplain{}{\thepage}}
\rfoot[\fancyplain{}{}]%
     {\fancyplain{}{\scriptsize}}

%%%%%%%%%%%%%%%%%%%%%%%%%%%%%%%%%%%%%%%%%%%%%%%%%%%%%%%%%%%%%%%%%%%%%%%%%%
%%%%%                      Start part here                          %%%%%%
%%%%%%%%%%%%%%%%%%%%%%%%%%%%%%%%%%%%%%%%%%%%%%%%%%%%%%%%%%%%%%%%%%%%%%%%%%

\lettrine[lines=1]{S}{ } hould I start this by declaring that these PhD years have been alternately depressing and engaging, exhausting and stimulating, infuriating and enthralling? This is trite, and true for everyone working through any kind of commitment, PhD student or not. Covid pandemic or not.  The birth of a child or not. Struggles in close friends’ and relatives’ lives or not. But there it is. Now that I’ve said so, let me go straight to my acknowledgements.

Above anyone else, I want to thank my mother. She not only had to deal with the difficult task of raising me and putting up with my constant flow of questions, but also with welcoming the four younger sisters that came after me. As a widow. With debts to pay off, and very little money coming in. Moving every two years, until we settled for a small apartment in a neighborhood that some would call a ghetto, although we preferred calling it home. And yet, there was always food on the table. Even better, we had no idea how poor we were, because she literally sacrificed her life for ours, and her passions for our interests. This is quintessential Christlike love. We all had the incredible opportunity of doing at least one physical and one artistic activity, on top of pursuing university level studies. We also learned how to live happily with very little, which I’m starting to realize is a sort of superpower. Most importantly, she raised children that all love each other. Now that I’m a father too, I can measure how high she set the bar, and I can only hope to be half as good as her. I can’t award her the Legion of Honor she deserves, but at least here is a little bit of recognition! Thank you from all of us, Maman.

I also have deep feelings for my father, who tragically passed away when I was still a small child. He did have to struggle with some issues that would eventually cause his death, but I believe he fought until the very end. He once taught me an important lesson in persistence, probably without even realizing it. A friend and I were racing up a hill, while my father timed us. I lost. We raced again; I lost again. I kept trying, and sure enough, I lost every single race. I went to my dad and complained, "I’m tired Papa, can we stop?" "Are you tired, really? Very good, it means that you’re on your way to making progress!" I paused, and let that sink in for a few moments. And without a word, I went back running. That’s how I learned that getting better goes with accepting a little suffering. Later, I also realized that out of any bad experience, even death, you can take away something positive, something that will help you grow. Against all odds, after having lost all these races, I even made my first professional career in sports. I am very grateful for both my parents: I am who I am, with all my quirks and all that’s to be loved or hated, thanks to them.

So many more people to thank! I’m just getting started, so you will have to bear with me for a little while yet. Let’s start with the sisters. Esther comes just after me; she married an awesome guy from Congo and is currently raising two wonderful little girls. She is the closest to what my mom was for us (and still is), making anyone feel at home at any time, always on the move, taking care of her family during the day and working nights, juggling countless tasks and thinking it’s all just natural. Then comes Déborah, although she didn’t come alone since Joëlla followed 10 minutes later. But believe it or not, she is a complete person all by herself. She has a high sense of justice and a desire to be helpful, which made her switch from the art history field to the mentally challenging field of health. Joëlla is also incredible. She fights her own health issues every day, could not finish high school but still managed to get a bachelor’s degree a few years later, and is now a professional violinist, whose empathy shines through all her music. I’m on a roll now, and I don’t think you’ll be surprised if I tell you that my last sister, Noémie, is more than decent. She also became a professional violinist, she runs every day, and she is currently studying psychology. And since this is not enough to deplete her energy levels, she has also been known to spend whatever is left on mediating arguments between people she loves. A family I’m proud of, not only because of their talents, but because of their virtues.

I want to thank my grandparents, whose house was home base for all of my sisters, cousins, aunts, and uncles on each and every vacation. They made us discover the delightful joy of being cold, wet, and exhausted during rainy hikes, to finally end up above a splendid sea of cloud from which would protrude just a few sharp peaks, over which Alpine choughs skillfully maneuvered in vigorous flight. The grandparents are the true pillars of our extended family. The cycle of life being what it is, they are becoming older and can’t hike anymore. I am now very happy to see the whole family striving to take care of them, as much as we have been taken care of. I sadly cannot name every single other member of my family, human or animal, but they are all a crucial part of myself.

I do need to spend some time on the love of my life, Mikaela. We met in Lebanon, she is American, she cares about France as little as I care about the USA, and yet she agreed to come here for me, in a strange country far from home. She had the courage to take over my mother’s difficult job of bearing with my incessant questions – and she actually has a lot of answers, since the extent of her knowledge is so wide. Mikaela is also an awesome writer and a qualified editor who plays a large role in making my productions publishable. She would like it stated, however, that any mistakes in what follows are most definitely not her fault. She is much more than she believes of herself: exceedingly faithful, remarkably generous, paradoxically very introverted but willing to welcome anyone in need whom she meets, and unfortunately suffering from how little power she has to make the world a better place. She also comes with a very nice family (her mother also spent some time correcting my English here), and of course, she is the mother of my child Cédric! A stunning baby who spends an excessive amount of energy smiling at everyone, all day long (aside from the occasions when he’s screaming his head off). He might give me a hard time whenever I get started writing my dissertation, but he does it in a very cute way. And he always provides a good excuse for our shared and legendary absent-mindedness. I’m looking forward to the time I’ll be old enough for him to change my own diapers.

Life wouldn’t be life without friends, old and new, whether I see them several times a week or once every two or three blue moons. Friends of the family, friends from church, friends from parkour, friends from the performing world, friends I have no idea how I met. Not to brag, but they are too numerous to name them all.

Finally, let’s remember that this is a PhD thesis I’m writing, and there’s no thesis without a lab, without supervisors, without fellow PhD students, post-docs, interns, researchers, administrative workers, cleaning operatives, and all who are involved in making work enjoyable (yes, I really said enjoyable). I want to thank them all. Lionel, my director, who saved me from the happy hell of being a starving performing artist to give me the chance to throw myself into another precariously fun career path. The subject of my dissertation could not be better suited to my aspirations: both closely related to sports, and highly technical. Mathieu, my co-supervisor, who always made himself available, ready to give me quick and valuable feedback at any time, and to come to Grenoble in person despite living at the other end of the country. One supervisor is an expert in computer vision, the other in biomechanics: the perfect fit for the objectives of my doctorate. Thibault, my faithful and multitasking office colleague, whom I often left alone with the cold, impersonal computer hardware while I worked remotely. Other colleagues from places such as the INSEP, the LBMC, the Pprime institute, etc. Athletes and coaches, who offered us some of their precious training time, knowing full well that our research would most likely benefit the next generation, rather than theirs. Thank you all!

To sum it up, I owe this work to my family, my friends, my colleagues, and I’m foolish enough to believe I owe it to God above all. I am happy I have overcome this challenge, not only alone but with all the aforenamed people!

And with these words, I suppose I can now start on what I’m here for.

\begin{textblock*}{10cm}(5cm,30.3cm) % {block width} (coords left,top) 
      \begin{turn}{0} 
      \scriptsize Also, to Fyodor Dostoevsky, who wrote “The Gambler” in 27 days in a state of pure panic, and whom I now fully understand. \newline
      \end{turn}
\end{textblock*}

\begin{textblock*}{10cm}(20cm,20cm) % {block width} (coords left,top) 
      \begin{turn}{90} 
            \normalsize \emojiegg
            \scriptsize Find 9 hidden easter eggs! 
      \end{turn}
\end{textblock*}