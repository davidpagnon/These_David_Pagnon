%%%%%%%%%%%%%%%%%%%%%%%%%%%%%%%%%%%%%%%%%%%%%%%%%%%%%%%%%%%%%%%%%%%%%%%%%%%%
%%%%%                          GLOSSAIRE                              %%%%%%
%%%%%%%%%%%%%%%%%%%%%%%%%%%%%%%%%%%%%%%%%%%%%%%%%%%%%%%%%%%%%%%%%%%%%%%%%%%%

\phantomsection 
\addcontentsline{toc}{chapter}{Glossary} \mtcaddchapter
\label{Ann:gloss}
\addtocontents{toc}{\protect\addvspace{5pt}}

\vspace*{-1.6cm}
\begin{flushright}
\section*{\fontsize{20pt}{20pt}\selectfont\textnormal{Glossary}}
\end{flushright}
\vspace{-0.2cm}


\lhead[\fancyplain{}{Glossary}]
      {\fancyplain{}{}}
\chead[\fancyplain{}{}]
      {\fancyplain{}{}}
\rhead[\fancyplain{}{}]
      {\fancyplain{}{Glossary}}
\lfoot[\fancyplain{}{}]
      {\fancyplain{}{}}
\cfoot[\fancyplain{}{\thepage}]
      {\fancyplain{}{\thepage}}
\rfoot[\fancyplain{}{}]%
     {\fancyplain{}{\scriptsize}}

%%%%%%%%%%%%%%%%%%%%%%%%%%%%%%%%%%%%%%%%%%%%%%%%%%%%%%%%%%%%%%%%%%%%%%%%%%
%%%%%                      Start part here                          %%%%%%
%%%%%%%%%%%%%%%%%%%%%%%%%%%%%%%%%%%%%%%%%%%%%%%%%%%%%%%%%%%%%%%%%%%%%%%%%%

\lettrine[lines=1]{S}{ }ome terms used along this thesis may lead to confusion to an uninitiated person. Some expressions can be closely related, but describe different concepts. This glossary compares and clarifies them.

\vspace*{1cm}

\noindent\textbf{Markerless vs. Sensorless. }
Camera sensors are not 


\vspace*{0.5cm}

\noindent\textbf{Markers vs. Keypoints. }

\vspace*{0.5cm}

\noindent\textbf{Silhouette vs. Shape. }
morphology
vs. mesh (geometric representation of a shape)

\vspace*{0.5cm}

\noindent\textbf{Gold standard vs. Silver standard}

\vspace*{0.5cm}

\noindent\textbf{Machine learning vs. Deep learning. }
vs. IA vs. Neural networks
cf https://www.bbntimes.com/science/artificial-intelligence-vs-machine-learning-vs-artificial-neural-networks-vs-deep-learning
data-driven approaches (vs. knowledge-driven)

\vspace*{0.5cm}

\noindent\textbf{Kinematics vs. Kinetics. }
vs dynamics

\vspace*{0.5cm}

\noindent\textbf{Spatio-temporal parameters vs Joint kinematics. }
both fall under the umbrella of kinematic data. In the customary usage, kinematics describes joint kinematics.

\vspace*{0.5cm}

\noindent\textbf{Forward kinematics vs. Direct kinematics. }
vs. inverse kinematics




% top-down vs. bottom-up
% Monocular/single view vs. Multiview: videos or images

