% %%%%%%%%%%%%%%%%%%%%%%%%%%%%%%%%%%%%%%%%%%%%%%%%%%%%%%%%%%%%%%%%%%%%%%%%%%
% %                            PAQUETS USUELS                              %
% %                                                                        %
% %%%%%%%%%%%%%%%%%%%%%%%%%%%%%%%%%%%%%%%%%%%%%%%%%%%%%%%%%%%%%%%%%%%%%%%%%%

\usepackage[T1]{fontenc}
\usepackage[utf8]{inputenc}
\usepackage{indentfirst}
%\usepackage[francais]{babel}
\usepackage[english]{babel}  % francais = french babel
\usepackage{meta-donnees} % page de garde
\usepackage{meta-donnees2} %page de 4ième de couv
\usepackage{amssymb}
\usepackage{verbatim}
\usepackage{array}	
\usepackage{color}
\usepackage{cite}
% \usepackage{refcheck} %nécessite \nocite{*} à la fin du tex
\usepackage{xfrac}
\usepackage{mathptmx}
\usepackage{enumitem}
\usepackage{esvect}
\usepackage[squaren,Gray]{SIunits}
\usepackage{sistyle}
\usepackage{eurosym}  %pour obtenir le symbole Euro
% \usepackage{gensymb} %pour obtenir le symbole \degree
\usepackage{calligra}

\usepackage{animate}

\usepackage{blindtext}
\usepackage{tabto}

%%%%%%%%%%%%%%%%%%%%%%%%%%%%%%%%%%%%%%%%%%%%%%%%%%%%%%%%%%%%%%%%%%%%%%%%%%
%%%%%           Packages pour les entetes et pied de page           %%%%%%
%%%%%%%%%%%%%%%%%%%%%%%%%%%%%%%%%%%%%%%%%%%%%%%%%%%%%%%%%%%%%%%%%%%%%%%%%%

%\usepackage{picins}
\usepackage{fancyhdr}
%\usepackage{psboxit}  %%% A Inserer avant babel !!!! 
\usepackage{pifont}


%%%%%%%%%%%%%%%%%%%%%%%%%%%%%%%%%%%%%%%%%%%%%%%%%%%%%%%%%%%%%%%%%%%%%%%%%%
%%%%%                   Packages et couleurs perso                  %%%%%%
%%%%%%%%%%%%%%%%%%%%%%%%%%%%%%%%%%%%%%%%%%%%%%%%%%%%%%%%%%%%%%%%%%%%%%%%%%

\usepackage{xcolor}  %%% Incompatibilité avec \usepackage{colortbl} ??????
\definecolor{BleuCyan}{RGB}{0,190,190}  %%% Définition d'une couleur personnelle
\definecolor{RoseRose}{RGB}{238,44,44}
\definecolor{VertVert}{RGB}{10,255,118}
\definecolor{Anthracite}{RGB}{91,124,151}
\definecolor{GrisPasTropClair}{RGB}{83,135,135}
\definecolor{BleuPetrole}{RGB}{0, 0 ,205}
\definecolor{BleuClair}{RGB}{234, 255 ,255}
\definecolor{Bleu1}{RGB}{26, 64 ,145}
\definecolor{Rouge1}{RGB}{215, 19 ,24}
\definecolor{myblue}{rgb}{.8, .8, 1}
\definecolor{violet1}{rgb}{0.78,0.53,0.97}
\definecolor{myblue2}{rgb}{0,0.41,0.54} % dark blue
\definecolor{myred}{RGB}{192,0,0} % dark red
\definecolor{mygreen2}{RGB}{0,120,0} % dark green
\definecolor{myblue3}{HTML}{1a4091} % dark blue 	
\definecolor{couleur_marie}{RGB}{0,102,204} % dark green

%\newcommand*\maboite[1]{%
%\fcolorbox{GrisPasTropClair}{BleuClair}{\hspace{1em}#1\hspace{1em}}}
%\newcommand{\parttoccolor}{blue}
%\newcommand{\chaptertoccolor}{red}
%\newcommand{\sectiontoccolor}{green!70!black}



%%%%%%%%%%%%%%%%%%%%%%%%%%%%%%%%%%%%%%%%%%%%%%%%%%%%%%%%%%%%%%%%%%%%%%%%%%
%%%%%            Packages pour les captions optimisés               %%%%%%
%%%%%%%%%%%%%%%%%%%%%%%%%%%%%%%%%%%%%%%%%%%%%%%%%%%%%%%%%%%%%%%%%%%%%%%%%%
\usepackage[font=small,font={it}]{caption} % \usepackage[small,hang]{caption2} apparement cpation2 est obsolette
%\captionsetup[table]{position=bottom}
%\renewcommand{\captionfont}{\it \small}
%\renewcommand{\captionlabelfont}{\it \bf \small}
% \renewcommand{\captionlabeldelim}{ :}  % Ne marche qu'avec caption2



%%%%%%%%%%%%%%%%%%%%%%%%%%%%%%%%%%%%%%%%%%%%%%%%%%%%%%%%%%%%%%%%%%%%%%%%%%
%%%%%           Packages pour les titres de chapitres               %%%%%%
%%%%%%%%%%%%%%%%%%%%%%%%%%%%%%%%%%%%%%%%%%%%%%%%%%%%%%%%%%%%%%%%%%%%%%%%%%
%\usepackage[Bjornstrup]{fncychap}
\usepackage[avantgarde]{quotchap}

\usepackage{lettrine}    %Lettrine exemple: \lettrine[lines=2]{L}{orem ipsum}
\usepackage[english]{minitoc}		% Pour ajouter une table des matières à chaque chapitre
\setcounter{minitocdepth}{2}
\AtBeginDocument{\def\chapterautorefname{Chapter}} % pour capitaliser les chapitres


%%%%%%%%%%%%%%%%%%%%%%%%%%%%%%%%%%%%%%%%%%%%%%%%%%%%%%%%%%%%%%%%%%%%%%%%%%
%%%%%                Packages pour les figures                      %%%%%%
%%%%%%%%%%%%%%%%%%%%%%%%%%%%%%%%%%%%%%%%%%%%%%%%%%%%%%%%%%%%%%%%%%%%%%%%%%
\usepackage{epsfig}
\usepackage{wrapfig}  %%% Inserer du texte a droite ou à gauche de l'image
%\usepackage{picins}  %%% Inserer du texte a droite ou à gauche de l'image
\usepackage{float}
%\usepackage{subfigure}
\usepackage{subcaption}
\usepackage[export]{adjustbox}
\usepackage{pgf}

\usepackage{pgfplots}
\usepgfplotslibrary{external}
\tikzexternalize
\usepackage{tikz}
\usetikzlibrary{positioning,backgrounds,fadings,shadows.blur,shadows}
\usetikzlibrary{fit,shapes.misc}

%%%%%%%%%%%%%%%%%%%%%%%%%%%%%%%%%%%%%%%%%%%%%%%%%%%%%%%%%%%%%%%%%%%%%%%%%%
%%%%%                  Packages pour les tableaux                   %%%%%%
%%%%%%%%%%%%%%%%%%%%%%%%%%%%%%%%%%%%%%%%%%%%%%%%%%%%%%%%%%%%%%%%%%%%%%%%%%
\usepackage{array}
\usepackage{textcomp}
\usepackage{booktabs}
\usepackage{colortbl}  %%% Couleurs de cellule de tableau
\usepackage{longtable}
\usepackage{lscape}    %%% Rotation des tableaux
\usepackage{multirow}
\usepackage{tabularx}
\usepackage{diagbox}
\usepackage{pifont}
\usepackage{vcell}
%\usepackage[table]{xcolor}

\usepackage{colortbl}

\arrayrulecolor{myblue3}
\let\oldtabular=\tabular
\def\tabular{\small\oldtabular}

%%%%%%%%%%%%%%%%%%%%%%%%%%%%%%%%%%%%%%%%%%%%%%%%%%%%%%%%%%%%%%%%%%%%%%%%%%
%%%%%                  Packages pour les maths                  %%%%%%
%%%%%%%%%%%%%%%%%%%%%%%%%%%%%%%%%%%%%%%%%%%%%%%%%%%%%%%%%%%%%%%%%%%%%%%%%%
\usepackage{amsmath}
\usepackage{tcolorbox}
\tcbuselibrary{/tcb/library/breakable}
% \usepackage{breqn} % Pour separer les equations sur plusieurs lignes automatiquement
\usepackage{algcompatible}
\usepackage{algorithm}
\usepackage{bm}

%\usepackage{empheq} % Pour encadrer les equations
%pour utiliser plusieurs fichiers bibteX
%\usepackage{biblist}
\usepackage{url} %to display url in bibliography with howpublished = {\url{blabla}}

% \usepackage{listings} % pour les code blocks
\usepackage{minted} % difficult to install in vscode
\AtBeginEnvironment{minted}{\setlength{\parskip}{0pt}}

%\usepackage{fourier}
\usepackage{SIunits}

%%%%%%%%%%%%%%%%%%%%%%%%%%%%%%%%%%%%%%%%%%%%%%%%%%%%%%%%%%%%%%%%%%%%%%%%%%
%%%%%              Packages pour les liens hypertexte               %%%%%%
%%%%%%%%%%%%%%%%%%%%%%%%%%%%%%%%%%%%%%%%%%%%%%%%%%%%%%%%%%%%%%%%%%%%%%%%%%
\usepackage{hyperref} %%%  Packages pour les liens hypertexte a mettre après tous les autres packages
 \hypersetup{
 colorlinks=true,
 %linkcolor=BleuCyan,
 linkcolor=couleur_marie,
 citecolor=BleuPetrole,
 filecolor=VertVert
 }
 
%  hyperindex=true,