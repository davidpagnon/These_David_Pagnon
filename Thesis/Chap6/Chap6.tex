%%%%%%%%%%%%%%%%%%%%%%%%%%%%%%%%%%%%%%%%%%%%%%%%%%%%%%%%%%%%%%%%%%%%%%%%%%
%%%%%                         CHAPITRE 6                            %%%%%%
%%%%%%%%%%%%%%%%%%%%%%%%%%%%%%%%%%%%%%%%%%%%%%%%%%%%%%%%%%%%%%%%%%%%%%%%%%

\lhead[\fancyplain{}{\leftmark}]%Pour les pages paires \bfseries
      {\fancyplain{}{}} %Pour les pages impaires
\chead[\fancyplain{}{}]%
      {\fancyplain{}{}}
\rhead[\fancyplain{}{}]%Pour les pages paires 
      {\fancyplain{}{\rightmark}}%Pour les pages impaires \bfseries
\lfoot[\fancyplain{}{}]%
      {\fancyplain{}{}}
\cfoot[\fancyplain{}{\thepage}]%\bfseries
      {\fancyplain{}{\thepage}} %\bfseries
\rfoot[\fancyplain{}{}]%
     {\fancyplain{}{\scriptsize}}


%%%%%%%%%%%%%%%%%%%%%%%%%%%%%%%%%%%%%%%%%%%%%%%%%%%%%%%%%%%%%%%%%%%%%%%%%%
%%%%%                      Start part here                          %%%%%%
%%%%%%%%%%%%%%%%%%%%%%%%%%%%%%%%%%%%%%%%%%%%%%%%%%%%%%%%%%%%%%%%%%%%%%%%%%

\chapter{Application to boxing, using action cameras}
\label{ch:6}

%==============================================================================	Résumé du chapitre

\begin{center}
\rule{0.7\linewidth}{.5pt}
\begin{minipage}{0.7\linewidth}
\smallskip

\textit{Pose2Sim in suboptimal conditions: \newline \newline
This chapter is adapted from the poster presented at the congress of the Europen College of Sport Science (ECSS): "A 3D markerless protocol with action cameras – Key performance indicators in boxing" \cite{Pagnon2022c}.
}

%\smallskip
\end{minipage}
\smallskip
\rule{0.7\linewidth}{.5pt}
\end{center}

\minitoc
\newpage


Calibration remains a challenging task in daylight, at a distance, with non research-grade cameras, and in a sports scene. It could be useful to make it more robust, either by implementing the Aniposelib library \cite{Karashchuk2020}, or by calibrating automatically on people’s limb length \cite{Liu2022a}.

Along with synchronization, this topic will be detailed in Chapter 6 on \nameref{ch:6}.

La calibration sera impossible si vous êtes trop loin. Les marqueurs réfléchissants ne réfléchiront pas la lumière des caméras (à tester : marqueurs actifs).  
Si vous voulez effectuer une post-calibration avec un checkerboard (avec cette méthode par exemple), il faudra que le checkerboard soit assez grand pour qu’il soit bien détecté. Une règle simple : pour avoir de bons résultats, la largeur du checkerboard doit remplir au moins un cinquième de l’image. Si vous voulez couvrir une scène de 20 m, il faudra un checkerboard de 4 m de large...  
Autre solution non testée : Calculer hors manip les paramètres intrinsèques des caméras vidéo. En manip, placer côte à côte des caméras MoCap Arqus et vidéo Miqus, faire la calibration des Arqus (plus performantes), et ajouter une translation dans les paramètres extrinsèques des Arqus pour avoir ceux des Miqus. 


\section{Objectives}
\subsection{Key Performance Indicators in boxing}
\blindtext

\subsection{Limits of research-grade systems in competitions}
\blindtext

\subsection{Objectives}
\blindtext


\section{Methods}
\subsection{4 conditions}
\blindtext

\subsection{Pose-calibration on ring dimensions}
\blindtext

\subsection{Post-synchronization on 2D movement speeds}
\blindtext

\subsection{GoPro spatio-temporal base into Qualysis'}
\blindtext

\subsection{Statistical analysis}
\blindtext


\section{Results}
\blindtext


\section{Discussion}
\subsection{Equipment and protocol vs. pose estimation model}
\blindtext

\subsection{Pros and cons of different systems}

Auto-calibration with person?

Cloud computing?

Temporal consistency?

Shape information for less cameras?

\blindtext
