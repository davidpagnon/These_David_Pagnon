%%%%%%%%%%%%%%%%%%%%%%%%%%%%%%%%%%%%%%%%%%%%%%%%%%%%%%%%%%%%%%%%%%%%%%%%%%
%%%%%                        Remerciements                          %%%%%%
%%%%%%%%%%%%%%%%%%%%%%%%%%%%%%%%%%%%%%%%%%%%%%%%%%%%%%%%%%%%%%%%%%%%%%%%%%

\phantomsection 
\addcontentsline{toc}{section}{Acknowledgements}
\addtocontents{toc}{\protect\addvspace{5pt}}

\vspace*{-1.6cm}
\begin{flushright}
\section*{\fontsize{20pt}{20pt}\selectfont\textnormal{Acknowledgements}}
\end{flushright}
\vspace{-0.2cm}


\lhead[\fancyplain{}{Acknowledgements}]
      {\fancyplain{}{}}
\chead[\fancyplain{}{}]
      {\fancyplain{}{}}
\rhead[\fancyplain{}{}]
      {\fancyplain{}{Acknowledgements}}
\lfoot[\fancyplain{}{}]
      {\fancyplain{}{}}
\cfoot[\fancyplain{}{\thepage}]
      {\fancyplain{}{\thepage}}
\rfoot[\fancyplain{}{}]%
     {\fancyplain{}{\scriptsize}}

%%%%%%%%%%%%%%%%%%%%%%%%%%%%%%%%%%%%%%%%%%%%%%%%%%%%%%%%%%%%%%%%%%%%%%%%%%
%%%%%                      Start part here                          %%%%%%
%%%%%%%%%%%%%%%%%%%%%%%%%%%%%%%%%%%%%%%%%%%%%%%%%%%%%%%%%%%%%%%%%%%%%%%%%%

\lettrine[lines=1]{S}{ } hould I start this by declaring that these PhD years have been alternatively depressing and engaging, exhausting and stimulating, infuriating and enthralling? This is trite, and true for everyone working through any kind of commitment, PhD student or not. Covid pandemic or not. Child birth or not. Struggles in close friends' and relatives' lives or not. But there it is. Now that it is stated, let me go straight to my acknowledgements.

Above anyone else, I want to thank my mother. She not only had to deal with the difficult task of raising me and putting up with my constant flow of questions, but also with welcoming the four smaller sisters that came after me. As a widow. With debts to pay off, and very little money coming in. Moving every two years, until we settled in for a small apartment in a neighborhood that some would call a ghetto, although we preferred calling it home. And yet, there was always food on the table. Even better, we had no idea how poor we were, because she literally sacrificed her life for ours, and her passions for our interests. This is quintessential Christlike love. We all had the incredible opportunity of doing at least one physical, and one artistic activity, on top of pursuing university level studies. We also learned how to live happily with very little, which I'm starting to realize is a sort of superpower. Most importantly, she made children that all love each other. Now that I'm a father too, I can measure how high she set the bar, and I can only hope to be half as good as her. I can't award her the Legion of Honor she deserves, but at least here is a little bit of recognition! Thank you from all of us, maman.

I also have a deep thought for my father, who tragically passed away when I was still a little child. He did have to struggle with some issues that would eventually cause his death, but I believe he fought until the very end. He is actually the one who taught me a nice lesson of persistence, surely without even trying. A friend and I were racing up a hill, while my father timed us. I lost. We raced again, I lost again. I tried more, and sure enough, I lost every single race. I went to my dad and complained: "I'm tired papa, can we stop?" "Are you tired, really? Very good, it means that you're on your way to make progress!" I paused, and let it sink in for a few moments. And without a word, I went back running. That's how I learned that getting better goes with accepting to suffer a little. Later on, I also realized that out of any bad experience, be it death, you can take away something positive, something that will help you grow. Against all odds, I even made a first professional carrier in sports. I am very grateful for both my parents: I am who I am, with all my quirks and all that's to be loved or to be hated, thanks to them.

So many more people to thank! I'm just getting started, so you will bear with me for a little while, still. But let's start with the sisters. Esther comes just after me, she married an awesome guy from Congo, and is currently raising two wonderful little girls. She is the closest to what my mom was with us (and still is), making anyone feel home at any time, always on the move, taking care of her family during the day and working at nights, juggling countless tasks and thinking it is all just natural. Then comes Déborah, although she didn't come alone since Joëlla followed 10 minutes later. But believe it or not, she is slightly more than a twin. She has a high sense of justice and a desire to be helpful, which made her switch from the arts history field to the mentally challenging health one, so as to be more true to herself. Joëlla also is incredible. She fights every day her own health issues, could not finish high school but still managed to get a bachelor degree a few years later, and she now is a professional violinist, whose empathy perspires through all her plays. I'm on a roll now, and I don't think you'll be surprised if I tell you that my last sister, Noémie, is decent enough. She also became a professional violinist, she runs every day, and she is currently studying psychology. She also spends a lot of energy mediating arguments between people she loves. A family I'm proud of, not only because of their obvious skills, but because of their virtues.

I want to thank my grandparents, whose house was the ground base for all of my aunts, uncles, and cousins, who met there during each and every vacation. They made us discover the delightful joy of being cold, wet and exhausted during rainy hikes, to finally end up above a splendid sea of cloud from which would protrude just a few sharp peaks, over which Alpine choughs skillfully maneuvered with their vigorous flight. The grandparents are the true pillars of our extended family. The cycle of life being what it is, they are becoming older and can't hike anymore. I am now very happy to see the whole family striving to take care of them, as much as we have been taken care of. I can sadly not name every single other member of my family, humans or animals, but they are all a crucial part of myself. 

I do need to spend some time for the love of my life, Mikaela. We met in Lebanon, she is American, she cares about France as little as I care about the USA, and yet she accepted to come here for me, in the armpit of the stinky old world. She had the courage to take over my mother's difficult job to bear with my incessant questions. She actually has a lot of answers, since the extent of her knowledge is so wide and well-rounded. Mikaela is also an awesome writer who regularly wins writing contests, and a qualified editor who plays a large role in making my productions publishable. She is much more than she believes of herself: exceedingly faithful, remarkably generous, paradoxically very introverted but willing to home all the persons in need she comes across, and unfortunately suffering from how little her power is to make the world a better place. She also comes with a very nice family in law, and of course, she is the mother of my child Cédric! A stunning baby who spends an excessive amount of energy smiling at every one, all day long (aside from sometimes, when he screams his head off.) He might give me a hard time whenever I get started writing my thesis, but he does it in a very cute way. And he always embodies a very good way for us to get away with our shared and legendary absent-mindedness. I'm looking forward to the time I'll be old enough for him to change my own diapers.

Life wouldn't be life without friends, old and new ones, whether I see them several times a week or once every two or three blue moons. Friends of the family, friends from church, friends from parkour, friends from the performing world, friends I have no idea how I got to know them. Not to brag, but they are too numerous to name them all. 

Finally, let's remember that this is a PhD thesis that I'm writing, and that there is no thesis without a lab, without supervisors, without fellow PhD students, post-docs, interns, researchers, administrative workers, cleaning operatives, and all who are involved in making work enjoyable (sic.) I want to thank them all. Lionel, my director, who saved me from the happy hell of starving performing arts to give me the chance to throw myself in another highly precariously fun situation. The subject of my thesis could not be better suited to my aspirations: both tightly related to sports, and highly technical. Mathieu, my co-supervisor, who always made himself available, ready to give me quick and valuable feedback at any time, and to come in Grenoble in person despite he lived in the other end of the country. One supervisor is expert in computer vision, the other in biomechanics: the perfect fit for the objectives of my doctorate. Thibault, my faithful and multitasker office colleague, that I often left alone with the sole presence of cold-blooded computer hardware while I worked remotely. Other colleagues from other places such as the INSEP, the LBMC, the Pprime institute, etc. Athletes and coaches, who offered us some of their precious training time, while they knew full well that our research would most likely benefit the next generation, rather than theirs. Thank you all!

To sum it up, I owe this work to my family, my friends, my colleagues, and I'm gullible enough to believe I owe it to God above all. I am happy I have overcome it, not only alone but with all the forenamed people!

On these words, I suppose I can now start with what I'm here for.