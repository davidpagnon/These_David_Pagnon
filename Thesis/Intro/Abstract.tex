%%%%%%%%%%%%%%%%%%%%%%%%%%%%%%%%%%%%%%%%%%%%%%%%%%%%%%%%%%%%%%%%%%%%%%%%%%
%%%%%                        Abstract                               %%%%%%
%%%%%%%%%%%%%%%%%%%%%%%%%%%%%%%%%%%%%%%%%%%%%%%%%%%%%%%%%%%%%%%%%%%%%%%%%%
\phantomsection 
\addcontentsline{toc}{section}{Abstract / Résumé}
\addtocontents{toc}{\protect\addvspace{5pt}}

\vspace*{-1.6cm}
\begin{flushright}
\section*{\fontsize{20pt}{20pt}\selectfont\textnormal{Abstract / Résumé}}
\end{flushright}
\vspace{-.2cm}

\lhead[\fancyplain{}{Abstract / Résumé}]
      {\fancyplain{}{}}
\chead[\fancyplain{}{}]
      {\fancyplain{}{}}
\rhead[\fancyplain{}{}]
      {\fancyplain{}{Abstract / Résumé}}
\lfoot[\fancyplain{}{}]%
      {\fancyplain{}{}}
\cfoot[\fancyplain{}{\thepage}]
      {\fancyplain{}{\thepage}}
\rfoot[\fancyplain{}{}]%
     {\fancyplain{}{\scriptsize}}
     

%%%%%%%%%%%%%%%%%%%%%%%%%%%%%%%%%%%%%%%%%%%%%%%%%%%%%%%%%%%%%%%%%%%%%%%%%%
%%%%%                      Start abstract here                      %%%%%%
%%%%%%%%%%%%%%%%%%%%%%%%%%%%%%%%%%%%%%%%%%%%%%%%%%%%%%%%%%%%%%%%%%%%%%%%%%

\vspace*{1cm}

\noindent\textbf{ABSTRACT: } \textsc{Design, evaluation, and application of a workflow for biomechanically consistent markerless kinematics in sports}

\lettrine[lines=1]{M}{ }otion capture is traditionally performed with marker-based systems. However, these solutions are hardly compatible with on-field sports analysis, and markerless alternatives are being explored. One of the most promising prospects lies at the intersection of machine learning for 2D pose estimation, computer vision for 3D reconstruction from multiple video sources, and biomechanics for constraining 3D coordinates to an anatomically consistent model. We proposed and released Pose2Sim, an open-source package striving to answer these needs in a user-friendly way. OpenPose 2D keypoint coordinates are robustly triangulated, and serve as input for a full-body OpenSim inverse kinematics procedure. Pose2Sim robustness has been evaluated, regarding people entering and exiting the field of view, degraded image quality, calibration errors, and low number of cameras. Its accuracy has also been assessed, and deemed sufficient for walking, running, and cycling analysis. In the context of a competition, using lightweight action cameras can be convenient. We tested such hardware on boxing sequences, and proposed post-calibration and post-synchronization procedures. Finally, capturing the equipment of the athlete along with them would be interesting. We explored the inverse kinematics of both a pilot and his bike in BMX race, by training a DeepLabCut bike model, triangulated and mapped on a custom articulated OpenSim model. This work brings out interesting new perspectives for the analysis of sports movement.

\vspace*{1cm}

\noindent\textbf{RÉSUMÉ : } \textsc{Conception, évaluation, et application d’une méthode biomécaniquement cohérente de cinématique sans marqueurs en sport}

\lettrine[lines=1]{L}{ }a capture de mouvement est traditionnellement effectuée à l'aide de marqueurs réfléchissants. Cependant, ces méthodes ne conviennent pas à l'analyse contextuelle du sport sur le terrain, et des alternatives sans marqueurs sont étudiées. L'une des perspectives les plus prometteuses à ce sujet se situe à l'intersection de l'apprentissage machine pour l'estimation de pose 2D, de la vision par ordinateur pour la reconstruction 3D à partir de plusieurs sources vidéo, et de la biomécanique pour contraindre les coordonnées 3D à un modèle anatomiquement cohérent. Nous avons proposé et publié Pose2Sim, un package open-source et simple d'utilisation visant à répondre à ces besoins. Les détections 2D d'OpenPose sont triangulées de manière robuste, et transmises à OpenSim pour une cinématique inverse corps complet. La robustesse de Pose2Sim a été estimée, face à des personnes "parasites" entrant le champ de vision, à une qualité d'image dégradée, à des erreurs de calibration, et à un nombre de caméras réduit. Son exactitude a également été évaluée, et jugée satisfaisante pour l'analyse de la marche, de la course, et du cyclisme. Dans un contexte de compétition, il peut être utile d'employer des caméras légères de type GoPro. Nous avons testé ce matériel sur des séquences de boxe, et proposé des procédures de post-calibration et de post-synchronisation. Enfin, capturer à la fois l'athlète et son équipement serait intéressant. Nous avons calculé la cinématique inverse d'un pilote de BMX avec son vélo, en entraînant un modèle DeepLabCut pour le vélo, triangulé et appliqué sur un modèle poly-articulé OpenSim. L'ensemble de ces résultats apporte des perspectives intéressantes et novatrices pour l'analyse du mouvement sportif.

\vspace*{1cm}
\noindent\textbf{KEYWORDS:} Markerless motion capture; Sports performance analysis; Kinematics; Computer vision; OpenPose; OpenSim; Python package






