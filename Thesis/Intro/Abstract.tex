%%%%%%%%%%%%%%%%%%%%%%%%%%%%%%%%%%%%%%%%%%%%%%%%%%%%%%%%%%%%%%%%%%%%%%%%%%
%%%%%                        Abstract                               %%%%%%
%%%%%%%%%%%%%%%%%%%%%%%%%%%%%%%%%%%%%%%%%%%%%%%%%%%%%%%%%%%%%%%%%%%%%%%%%%
\phantomsection 
\addcontentsline{toc}{section}{Abstract / Résumé}
\addtocontents{toc}{\protect\addvspace{5pt}}

\vspace*{-1.6cm}
\begin{flushright}
\section*{\fontsize{20pt}{20pt}\selectfont\textnormal{Abstract / Résumé}}
\end{flushright}
\vspace{-.2cm}

\lhead[\fancyplain{}{Abstract / Résumé}]
      {\fancyplain{}{}}
\chead[\fancyplain{}{}]
      {\fancyplain{}{}}
\rhead[\fancyplain{}{}]
      {\fancyplain{}{Abstract / Résumé}}
\lfoot[\fancyplain{}{}]%
      {\fancyplain{}{}}
\cfoot[\fancyplain{}{\thepage}]
      {\fancyplain{}{\thepage}}
\rfoot[\fancyplain{}{}]%
     {\fancyplain{}{\scriptsize}}
     

%%%%%%%%%%%%%%%%%%%%%%%%%%%%%%%%%%%%%%%%%%%%%%%%%%%%%%%%%%%%%%%%%%%%%%%%%%
%%%%%                      Start abstract here                      %%%%%%
%%%%%%%%%%%%%%%%%%%%%%%%%%%%%%%%%%%%%%%%%%%%%%%%%%%%%%%%%%%%%%%%%%%%%%%%%%

\vspace*{3cm}

\noindent\textbf{ABSTRACT: } \textsc{Design, evaluation, and application of a workflow for biomechanically consistent markerless kinematics in sports}

\lettrine[lines=1]{M}{ }arker-based motion capture is hardly compatible with the inherent constraints of sports. As a consequence, numerous markerless methods are being proposed. One of them, Pose2Sim, has been developed and published during this doctoral program. Pose2Sim bridges the most renown programs in their respective fields: OpenPose for 2D pose estimation, and OpenSim for physically consistent 3D kinematics. The robustness of this solution has been evaluated, as well as its accuracy. Pose2Sim has also been tested in more challenging experimental conditions, in spite of which boxing key performance indicators have been correctly evaluated. Finally, BMX start sequences have been analyzed. It has been shown that if the 2D pose estimator was good enough, it would be possible to use Pose2Sim to provide joint \{bike+pilot\} kinematics to coaches.

\vspace*{3cm}

\noindent\textbf{RÉSUMÉ : } \textsc{Conception, évaluation, et application d’une méthode biomécaniquement cohérente de cinématique sans marqueurs en sport}

\lettrine[lines=1]{L}{ }es contraintes inhérentes au mouvement sportif ne permettent généralement pas son analyse à l'aide de marqueurs. En conséquence, des méthodes sans marqueurs sont de plus en plus proposées. L'une d'entre elles, Pose2Sim, a été développée et publiée en open-source dans le cadre de cette thèse. Pose2Sim fait le lien entre les deux programmes les plus utilisés de leurs domaines respectifs : OpenPose pour l'estimation de pose 2D, et OpenSim pour la cinématique 3D physiquement réaliste. La robustesse de cette solution a été évaluée, ainsi que sa précision. Pose2Sim a aussi été mis à l'épreuve de conditions expérimentales plus complexes, malgré lesquelles les indicateurs de performance clé en boxe ont pu être correctement évalués. Enfin, des séquences de départ de BMX ont été analysées. Il a été démontré que si l'estimateur de pose 2D est assez performant, il est possible d'utiliser Pose2Sim pour fournir aux entraîneurs la cinématique de l'ensemble \{vélo+pilote\}.

\vspace*{3cm}
\noindent\textbf{KEYWORDS:} Markerless motion capture; Sports performance analysis; Kinematics; Computer vision; OpenPose; OpenSim; Python package






