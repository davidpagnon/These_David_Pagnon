%%%%%%%%%%%%%%%%%%%%%%%%%%%%%%%%%%%%%%%%%%%%%%%%%%%%%%%%%%%%%%%%%%%%%%%%%%
%%%%%                         CHAPITRE 2                            %%%%%%
%%%%%%%%%%%%%%%%%%%%%%%%%%%%%%%%%%%%%%%%%%%%%%%%%%%%%%%%%%%%%%%%%%%%%%%%%%

\lhead[\fancyplain{}{\leftmark}]%Pour les pages paires \bfseries
      {\fancyplain{}{}} %Pour les pages impaires
\chead[\fancyplain{}{}]%
      {\fancyplain{}{}}
\rhead[\fancyplain{}{}]%Pour les pages paires 
      {\fancyplain{}{\rightmark}}%Pour les pages impaires \bfseries
\lfoot[\fancyplain{}{}]%
      {\fancyplain{}{}}
\cfoot[\fancyplain{}{\thepage}]%\bfseries
      {\fancyplain{}{\thepage}} %\bfseries
\rfoot[\fancyplain{}{}]%
     {\fancyplain{}{\scriptsize}}


%%%%%%%%%%%%%%%%%%%%%%%%%%%%%%%%%%%%%%%%%%%%%%%%%%%%%%%%%%%%%%%%%%%%%%%%%%
%%%%%                      Start part here                          %%%%%%
%%%%%%%%%%%%%%%%%%%%%%%%%%%%%%%%%%%%%%%%%%%%%%%%%%%%%%%%%%%%%%%%%%%%%%%%%%

\chapter{Theoretical framework}
\label{ch:2}

%==============================================================================	Résumé du chapitre

\begin{center}
\rule{0.7\linewidth}{.5pt}
\begin{minipage}{0.7\linewidth}
\smallskip

\textit{Résumé du chapitre possible ici.
}

%\smallskip
\end{minipage}
\smallskip
\rule{0.7\linewidth}{.5pt}
\end{center}

\minitoc
\newpage


\section{Pose detection}

\subsection{The problem of image recognition}

Cf mail machine learning starred.

2D pose detections of images -> detecting features

Dedicated algorithms vs deep learning \newline\newline


- deep learning vs CNN, AI, machine learning

- classification vs detection vs segmentation


\subsection{Timeline and principles}

History natural neuron and formal neuron (dates, names, comparison)

Timeline cahier jaune et wiki

S'inspirer de wikipedia (en, fr, timeline); 
S'inspirer du mail machine learning starred (exemple du réseau de neurones); 
S'inspirer du cahier jaune


% \subsection{Application to object detection and localization}
% À supprimer ?

\subsection{Pose detection: Different architectures, different models, different datasets}
\blindtext



\section{3D reconstruction}\label{sec:3D reconstruction}

While some approaches only rely on 2D pose estimation to infer 3D pose with another machine learning model, they are generally not considered to be sufficiently reliable. It is, then, important to use the input from several cameras, and to fuse their informations to obtain 3D coordinates.

\subsection{Pinhole camera model}
\blindtext

\subsection{Calibration}
\blindtext

\subsection{Triangulation}
\blindtext


\section{3D joint kinematics}
\subsection{Physically consistent model}
\blindtext

\subsection{Scaling}
\blindtext

\subsection{Inverse kinematics}
As opposed to forward kinematics \newline
Compare with 2D angles between 3 points \newline
Different methods (model based vs autres) for angles (cf mail starred)\newline




