%%%%%%%%%%%%%%%%%%%%%%%%%%%%%%%%%%%%%%%%%%%%%%%%%%%%%%%%%%%%%%%%%%%%%%%%%%
%                         FORMAT PERSONNALISES                           %
% %%%%%%%%%%%%%%%%%%%%%%%%%%%%%%%%%%%%%%%%%%%%%%%%%%%%%%%%%%%%%%%%%%%%%%%%



%%%%%%%%%%%%%%%%%%%%%%%%%%%%%%%%%%%%%%%%%%%%%%%%%%%%%%%%%%%%%%%%%%%%%%%%%%
%%%%%       Page setup (see page 204 of the Latex Companion)        %%%%%%
%%%%%%%%%%%%%%%%%%%%%%%%%%%%%%%%%%%%%%%%%%%%%%%%%%%%%%%%%%%%%%%%%%%%%%%%%%

% Sets the origin of the page to the top left corner
\setlength{\voffset}{-1in}
\setlength{\hoffset}{-1in}

% Parameters to modify:

% Width parameters
\setlength{\oddsidemargin}{3cm} %Valeur des marges pages impaires mettre 4cm
\setlength{\evensidemargin}{3cm} %Valeur des marges pages paires mettre 2cm
\setlength{\textwidth}{15cm} %Largeur du texte 15cm il reste donc 6cm pour les marges droite et gauches

% Hight parameters
%\setlength{\topmargin}{2cm}
\setlength{\topmargin}{1.3cm}
\setlength{\headheight}{28pt} %Permet de caser deux lignes de 12pt de haut dans l'entete
\setlength{\headsep}{0.5cm}
\setlength{\textheight}{24.5cm}
\setlength{\footskip}{1.3cm}

% Margin note parameters
\setlength{\marginparsep}{0cm}
\setlength{\marginparwidth}{0cm}
\setlength{\marginparpush}{0cm}

%%%%%%%%%%%%%%%%%%%%%%%%%%%%%%%%%%%%%%%%%%%%%%%%%%%%%%%%%%%%%%%%%%%%%%%%%%
%%%%%                      Your fancy heading                    %%%%%%%%%
%%%%%%%%%%%%%%%%%%%%%%%%%%%%%%%%%%%%%%%%%%%%%%%%%%%%%%%%%%%%%%%%%%%%%%%%%%
%\pagestyle{Glenn}
\pagestyle{fancy}			% Format de les lignes haut de page
%\renewcommand{\chaptermark}[1]{\markright{\chaptername\ \thechapter.\ #1}}
%\renewcommand{\sectionmark}[1]{\markright{\thesection.\ #1}{}}
\renewcommand{\chaptermark}[1]{\markboth{\chaptername\ \thechapter.\ #1}{}}
\renewcommand{\sectionmark}[1]{\markright{\thesection.\ #1}}
\lhead[\fancyplain{}{\leftmark}]%Pour les pages paires \bfseries
      {\fancyplain{}{}} %Pour les pages impaires
\chead[\fancyplain{}{}]%
      {\fancyplain{}{}}
\rhead[\fancyplain{}{}]%Pour les pages paires 
      {\fancyplain{}{\rightmark}}%Pour les pages impaires \bfseries
\lfoot[\fancyplain{}{}]%
      {\fancyplain{}{}}
\cfoot[\fancyplain{}{\thepage}]%\bfseries
      {\fancyplain{}{\thepage}} %\bfseries
\rfoot[\fancyplain{}{}]%
     {\fancyplain{}{\scriptsize}}


%% Draws a solid line at the head and foot of the document
\renewcommand\headrulewidth{1pt}
\renewcommand\footrulewidth{1pt}
%\renewcommand{\headrule}{{\color{myblue3}}}
%\renewcommand{\footrule}{{\color{myblue3}}}

% Change la couleur des lignes d'entete et de bas de page
%\renewcommand{\headrule}{{\color{violet1} \hrule width\headwidth height\headrulewidth \vskip-\headrulewidth}}
%\renewcommand{\footrule}{{\color{violet1} \hrule width\headwidth height\headrulewidth \vskip-\headrulewidth}}

%\newcommand{\parttoccolor}{blue}
%\newcommand{\chaptertoccolor}{red}
%\newcommand{\sectiontoccolor}{green!70!black}

%\makeatletter
%\renewcommand*\l@part[2]{%
%  \ifnum \c@tocdepth >-2\relax
%    \addpenalty{-\@highpenalty}%
%    \addvspace{2.25em \@plus\p@}%
%    \setlength\@tempdima{3em}%
%    \begingroup
%      \parindent \z@ \rightskip \@pnumwidth
%      \parfillskip -\@pnumwidth
%      {\leavevmode
%       \large \bfseries \color{\parttoccolor}#1\hfil
%       \hb@xt@\@pnumwidth{\hss
%       \def\@linkcolor{\parttoccolor}#2}}\par
%       \nobreak
%         \global\@nobreaktrue
%         \everypar{\global\@nobreakfalse\everypar{}}%
%    \endgroup
%  \fi}
%\renewcommand*\l@chapter[2]{%
%  \ifnum \c@tocdepth >\m@ne
%    \addpenalty{-\@highpenalty}%
%    \vskip 1.0em \@plus\p@
%    \setlength\@tempdima{1.5em}%
%    \begingroup
%      \parindent \z@ \rightskip \@pnumwidth
%      \parfillskip -\@pnumwidth
%      \leavevmode \bfseries
%      \advance\leftskip\@tempdima
%      \hskip -\leftskip
%      \color{\chaptertoccolor}#1\nobreak\
%       \leaders\hbox{$\m@th
%        \mkern \@dotsep mu\hbox{.}\mkern \@dotsep
%        mu$}\hfil\nobreak\hb@xt@\@pnumwidth{\hss
%        \def\@linkcolor{\chaptertoccolor}#2}\par
%      \penalty\@highpenalty
%    \endgroup
%  \fi}
%\def\@dottedtocline#1#2#3#4#5{%
%  \ifnum #1>\c@tocdepth \else
%    \vskip \z@ \@plus.2\p@
%    {\leftskip #2\relax \rightskip \@tocrmarg \parfillskip -\rightskip
%     \parindent #2\relax\@afterindenttrue
%     \interlinepenalty\@M
%     \leavevmode
%     \@tempdima #3\relax
%     \advance\leftskip \@tempdima \null\nobreak\hskip -\leftskip
%     {#4}\nobreak
%     \leaders\hbox{$\m@th
%        \mkern \@dotsep mu\hbox{.}\mkern \@dotsep
%        mu$}\hfill
%     \nobreak
%     \hb@xt@\@pnumwidth{\hfil\normalfont #5}%
%     \par}%
%  \fi}
%\renewcommand*\l@section{\color{\sectiontoccolor}\def\@linkcolor{\sectiontoccolor}\@dottedtocline{1}{1.5em}{2.3em}}
%\makeatother
%%%%%%%%%%%%%%%%%%%%%%%%%%%%%%%%%%%%%%%%%%%%%%%%%%%%%%%%%%%%%%%%%%%%%%%%%%
%%%%%        Here you set the space between the main text           %%%%%%
%%%%%                and the start of the footnote                  %%%%%%
%%%%%%%%%%%%%%%%%%%%%%%%%%%%%%%%%%%%%%%%%%%%%%%%%%%%%%%%%%%%%%%%%%%%%%%%%%
\addtolength{\skip\footins}{5mm}


% * Pour augmenter l'espace avant une section, apres le titre il suffit, par
% exemple, d'écrire, dans le préambule du document :

%%%% debut macro %%%%
\makeatletter
\renewcommand\section{\@startsection{section}{1}{\z@}% 
	{1cm \@plus -1ex \@minus -.2ex}%
	{2.3ex \@plus.2ex}%
	{\reset@font\Large\bfseries}}
\makeatother
%%%% fin macro %%%%

%%%%%%%%%%%%%%%%%%%%%%%%%%%%%%%%%%%%%%%%%%%%%%%%%%%%%%%%%%%%%%%%%%%%%%%%%%
%%%%%                     Tableaux                               %%%%%%%%%
%%%%%%%%%%%%%%%%%%%%%%%%%%%%%%%%%%%%%%%%%%%%%%%%%%%%%%%%%%%%%%%%%%%%%%%%%%
% Taille des lignes epaisses tableau ( pour toprule)
\setlength {\heavyrulewidth }{0.14 em}


%%%%%%%%%%%%%%%%%%%%%%%%%%%%%%%%%%%%%%%%%%%%%%%%%%%%%%%%%%%%%%%%%%%%%%%%%%
%%%%%                     Spacing for the lists                  %%%%%%%%%
%%%%%%%%%%%%%%%%%%%%%%%%%%%%%%%%%%%%%%%%%%%%%%%%%%%%%%%%%%%%%%%%%%%%%%%%%%
 %\FrenchItemizeSpacingfalse %%% permet d'avoir un espace entre la fin d'un paragraphe et le début de la liste (\topsep ne marche pas)
 %\FrenchListSpacingfalse


