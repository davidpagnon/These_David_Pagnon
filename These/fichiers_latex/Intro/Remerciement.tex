%%%%%%%%%%%%%%%%%%%%%%%%%%%%%%%%%%%%%%%%%%%%%%%%%%%%%%%%%%%%%%%%%%%%%%%%%%
%%%%%                        Remerciements                          %%%%%%
%%%%%%%%%%%%%%%%%%%%%%%%%%%%%%%%%%%%%%%%%%%%%%%%%%%%%%%%%%%%%%%%%%%%%%%%%%

\phantomsection 
\addcontentsline{toc}{section}{Acknowledgements}
\addtocontents{toc}{\protect\addvspace{5pt}}

\vspace*{-1.6cm}
\begin{flushright}
\section*{\fontsize{20pt}{20pt}\selectfont\textnormal{Acknowledgements}}
\end{flushright}
\vspace{-0.2cm}


\lhead[\fancyplain{}{Acknowledgements}]
      {\fancyplain{}{}}
\chead[\fancyplain{}{}]
      {\fancyplain{}{}}
\rhead[\fancyplain{}{}]
      {\fancyplain{}{Acknowledgements}}
\lfoot[\fancyplain{}{}]
      {\fancyplain{}{}}
\cfoot[\fancyplain{}{\thepage}]
      {\fancyplain{}{\thepage}}
\rfoot[\fancyplain{}{}]%
     {\fancyplain{}{\scriptsize}}

%%%%%%%%%%%%%%%%%%%%%%%%%%%%%%%%%%%%%%%%%%%%%%%%%%%%%%%%%%%%%%%%%%%%%%%%%%
%%%%%                      Start part here                          %%%%%%
%%%%%%%%%%%%%%%%%%%%%%%%%%%%%%%%%%%%%%%%%%%%%%%%%%%%%%%%%%%%%%%%%%%%%%%%%%

\lettrine[lines=1]{S}{ } hould I start this by declaring that these PhD years have been alternatively depressing and engaging, exhausting and stimulating, infuriating and enthralling? This is trite, and true for everyone, PhD student or not. Covid pandemic or not. Child birth or not. Struggles in close friends' and relatives' lifes or not. It is true, though, so there it is. Now that it is stated, let me go straight to my acknowledgements.

Above anyone else, I want to thank my mother. She not only had to deal with the difficult task of raising me and putting up with my constant flow of questions, but also with welcoming the four smaller sisters that came after. As a widow. With debts to pay off, and very little money coming in. Moving every two years, until we settled in for a small appartment in a neighborhood that some would call a ghetto, although we prefered calling it home. And yet, there was always food on the table. Even better, we had no idea how poor we were, because she literally sacrificed her life for ours, her passions for our interests. This is quintescential Christlike love. We all had the incredible opportunity of doing at least one physical, and one artistic activity, on top of pursuing university level studies. We also learned how to live happily with very little, which I'm starting to realize is an incredible superpower. Now that I'm a father too, I can measure how high she set the bar, and I can only hope to be half as good as her. Most importantly, she made children that all love each other unconditionally. I can't award her the Legion of Honor she would deserve, but at least here is a little bit of recognition! Thank you from all of us, mom.

I also have a deep thought for my father, who tragically passed away when I was still a little child. He did have to struggle with some issues that would eventually cause his death, but I believe he fought until the very end. He is actually the one who taught me a nice lesson of persistence, surely without even trying. A friend and I were racing up a hill, while my father timed us. I lost, every single time. I went to him and said: "I'm tired papa, can we stop?" "Are you? This is very good news, it means that you're going to get better soon!" I let it sink in for a few moments, and without a word, I went back running. That's when I learned that you have to accept to suffer a little bit in order to make progress. Against all odds, I even made a short professional carrier in sports. Later on, I also realized that out of any bad experience, be it death, you can take out something positive that will make you grow. I am very grateful for both my parents. I am who I am, with all my quirks and all that's to be loved or to be hated, thanks to them.

Thanks to my sisters, too. And to the rest of our family, and to my friends. And to all the people I met. And to every little thing I went through. But let's start with the sisters. Esther comes just after me, she married an awesome guy from Congo, and is currently raising two incredible little girls. She is the closest to what my mom was (and still is) with us, welcoming anyone at any time, always on the move, working at nights and taking care of her family during the day, juggling countless tasks and thinking it is all just natural. Then comes Déborah, although she didn't come alone since Joëlla followed 10 minutes later. But believe it or not, she is slightly more than a twin. She has a high sense of justice, which made her switch from studying arts history to working in the health field, so as to be more true to herself. Joëlla also is incredible, she fights every day through her health issues, could not finish high school but still managed to get a bachelor degree, and she now is a professionnal violinist, whose empathy perspires through all her plays. I'm on a roll now, and I don't think you'll be suprised if I tell you that my last sister, Noémie, is decent enough. She also became a professionnal violinist, she runs every day, and she is currently studying psychology. She also spends a lot of energy mediating arguments between people she loves. A family I'm proud of.

I want to thank my grand-parents, whose house was the ground base for all of my aunts, uncles, and cousins. They made us discover the joy in the pain of exhaustion, through hiking. The cycle of life being what it is, they became older and can't hike anymore, but I'm delighted to see the whole family striving to take care of them as much as they have been taken care of. I can sadly not name every single other member of my family, whether they are humans or animals, but they are a crucial part of myself.

I do need to spend some time for the love of my life, Mikaela. We met in Lebanon, she is American, she cares about France as little as I care about the USA, and yet she accepted to come her for me, in the armpit of the old and stinky world. She had the courage to take over my mother's difficult job to bear with my incessant questions. She actually has a lot of answers, since the extend of her knowledge is so wide and well-rounded. She is also an awesome writer, and a qualified editer who plays a large role in making my productions publishable. She is much more than she believes of herself: exceedingly faithful, generous, unfortunately suffering of how little her power is to make the world a better place. She also comes with a very nice family in law, and of course, she is the mother of my child Cédric! An incredible baby who spends all of his energy smiling at every one, all day long (aside from sometimes, when he screams his head off.) He might give me a hard time whenever I get started in writing my thesis, but he does it in a very cute way. And he always embodies a very good way for us to get away with our shared legendary absent-mindedness. I'm looking forward to the time I'll be old enough for him to change my diapers.

Life wouldn't be life without friends, old and new ones, the ones I see several times a week as the ones I see once every two or three blue moons. Friends of the family, friends from church, friends from parkour, friends from the performing world, friends I have no idea how I got to know them. Not to brag, but they are too numerous to name them all. 

Finally, let's remember that this is a PhD thesis that I'm writing, and that there is no thesis without a lab, without supervisors, without fellow PhD students, post-docs, interns, researchers, administrative workers, cleaning operatives, and all who are involved in making work enjoyable (sic). I want to thank them all. Lionel, my director, saved me from the happy hell of starving performing arts to give me the chance to throw myself in another highly precariously fun situation. Mathieu, my co-supervisor, was quite present and helpful, always ready to give me quick and valuable feedback, despite he lived in the other end of the country. Thibault, my faithful office colleague, that I often left alone with the sole presence of cold-blooded computer hardware while I worked remotely. Other colleagues from other places such as the INSEP, the LBMC, the Pprime institute, etc. Thank you all!

To sum it up, I owe this work to my family, my friends, my colleagues, and I'm guillible enough to believe I owe it to God above all. I am happy I have overcome it, not only alone but with all of the aforenamed people!

On these words, I believe I can now start with what I'm here for.